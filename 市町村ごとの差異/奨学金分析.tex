\documentclass{jsarticle}
\usepackage[margin = .7in]{geometry}
\usepackage[dvipdfmx]{graphicx}
\usepackage{listings}
\usepackage{amsmath}
\usepackage{bm}
\lstset{%
  language={python},
  basicstyle={\small},%
  identifierstyle={\small},%
  commentstyle={\small\itshape},%
  keywordstyle={\small\bfseries},%
  ndkeywordstyle={\small},%
  stringstyle={\small\ttfamily},
  frame={tb},
  breaklines=true,
  columns=[l]{fullflexible},%
  numbers=left,%
  xrightmargin=0zw,%
  xleftmargin=3zw,%
  numberstyle={\scriptsize},%
  stepnumber=1,
  numbersep=1zw,%
  lineskip=-0.5ex%
}



\begin{document}

\title{給付型奨学金の効果検証}
\author{Yuta Tokuda and Kei Ikegami}
\maketitle

\section{分析}
各市町村が個別に実施している奨学金制度の差異を用いて、給付型奨学金の有無などが進学に影響するかどうかなどを分析する。市ごとの奨学金制度とは市区町村が地元の学生を対象に募集をかけている奨学金制度のことを指す。貸与型・給付型の有無やその金額、市ごとの奨学金受給者数、もらえる所得の目安、JASSOと併用できるかどうかなどの性質が地域によって異なるため、本来の目的である「給付型奨学金が進学者数を増やすか否か」という問題に答えられると考えた。

\section{推定}
推定式は以下である。
\begin{align*}
y_{mt}&=\alpha_{c}\bf1 (municipality)+(\gamma+\delta ML_{mt})\bf1(loan_{mt})+(\eta+\theta MB_{mt})\\[10pt]& \quad\  \bf1(benefit_{mt})+\kappa\bf1(JASSO)+X_{mt}\beta+\epsilon_{mt}
\end{align*}
ここで、
	\begin{description}
	        \item[$y_{mt}$] $m$市の$t$年における進学率、進学者数、奨学金受給者数など
		\item[$\bf1 (municipality)$]  市町村ダミー
		\item[$ML_{mt}$]  $m$市の$t$年の貸与型金額
		\item[$\bf1(loan_{mt})$]  $m$市に$t$年に貸与型奨学金制度があるかのダミー
		\item[$MB_{mt}$]  $m$市の$t$年の給付型金額
		\item[$\bf1(benefit_{mt})$]  $m$市に$t$年に給付型奨学金制度があるかのダミー
		\item[$\bf1(JASSO)$]  jassoの奨学金と併用できるかダミー
	   \item[$X_{mt}$]  コントロール変数
	\end{description}
	である。
\section{問題点・疑問点}
\begin{itemize}
\item 給付型の有無や奨学金の額については、その市の財政状況などに依存して決定されることから、前年度の市のdemographicなデータから予測可能かどうかなどのチェックをおこなう
\item 奨学金制度は、市ごとに奨学金の受け入れ人数が決まっていることから、進学者としてふやせる人数に上限が存在する。これにより結果変数に進学者数などを用いた場合に制限付きの変数となってしまう。このような状況で上の回帰式は意味をなすか。
\item 市町村レベルの奨学金は受け入れ人数が多くても数十人であり進学者数全体から見ればかなり少ない。このような僅かな変動を捉えられるほどのデータ量があるか怪しい。
\end{itemize}
\end{document}